%%%%%%%%%%%%%%%%%%%%%%%%%%%%%%%%%%%%%%%%%%%%%%%%%%%%%%%%%%%%%%

% two types of boxes to use for highlighting things
\NewEnviron{solidbox}{
    \begin{center}
    \par
    \begin{tikzpicture}
    \node[rectangle,minimum width=0.85\textwidth] (m) {
        \begin{minipage}{0.8\textwidth}\BODY\end{minipage}
    };
    \draw (m.south west) rectangle (m.north east);
    \end{tikzpicture}
    \end{center}
}

\NewEnviron{dashedbox}{
    \begin{center}
    \par
    \begin{tikzpicture}
    \node[rectangle,minimum width=0.85\textwidth] (m) {
        \begin{minipage}{0.8\textwidth}\BODY\end{minipage}
    };
    \draw[dashed] (m.south west) rectangle (m.north east);
    \end{tikzpicture}
    \end{center}
}

%%%%%%%%%%%%%%%%%%%%%%%%%%%%%%%%%%%%%%%%%%%%%%%%%%%%%%%%%%%%%%

% some commonly used math commands
\DeclarePairedDelimiter\abs{\lvert}{\rvert}
\DeclarePairedDelimiter\norm{\lVert}{\rVert}
\DeclarePairedDelimiterX\braket[2]{\langle}{\rangle}{#1 , #2}

\newcommand{\nnorm}[1]{
    {\left\vert\kern-0.25ex\left\vert\kern-0.25ex\left\vert #1
    \right\vert\kern-0.25ex\right\vert\kern-0.25ex\right\vert}
}
\newcommand{\set}[2]{\mleft\{ #1 \,\middle|\, #2 \mright\}}

\newcommand{\restr}[2][{}]{\,{\big|}^{#1}_{#2}}
\newcommand{\defgl}{\mathrel{=\!\!\mathop:}}
\newcommand{\defgr}{\mathrel{\mathop:\!\!=}}
\newcommand{\C}{\mathbb{C}}
\newcommand{\R}{\mathbb{R}}
\newcommand{\Q}{\mathbb{Q}}
\newcommand{\Z}{\mathbb{Z}}
\newcommand{\N}{\mathbb{N}}
\newcommand{\K}{\mathbb{K}}
\newcommand{\HH}{\mathbb{H}}

\newcommand{\id}[1]{\mathrm{id}_{#1}}
\newcommand{\Id}[1]{\mathrm{Id}_{#1}}
\newcommand{\function}[3]{#1: #2 \rightarrow #3}

%%%%%%%%%%%%%%%%%%%%%%%%%%%%%%%%%%%%%%%%%%%%%%%%%%%%%%%%%%%%%%

% some optional packages
%\usepackage{graphicx}
%\usepackage{subcaption}
%\usepackage{showlabels}
%\usepackage{color}
%\definecolor{bg}{rgb}{0.95,0.95,0.95}